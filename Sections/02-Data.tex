\section{Data}
Dataset yang digunakan dalam penelitian ini adalah \textit{Student Performance Metrics Dataset} \textcite{Hasan2024}. Dataset ini diterbitkan tahun 2024 dan mencakup 493 observasi mahasiswa dengan 16 atribut, mencakup variabel demografis, prestasi akademik, kondisi sosial-ekonomi, serta kegiatan ekstrakurikuler. Berikut adalah atribut yang terdapat pada dataset ini:

\begin{itemize}
    \item \textbf{Department}: Jurusan akademik mahasiswa
    \item \textbf{Gender}: Jenis kelamin
    \item \textbf{HSC}: Skor ujian tingkat menengah atas
    \item \textbf{SSC}: Skor ujian tingkat menengah pertama
    \item \textbf{Income}: Pendapatan keluarga per bulan
    \item \textbf{Hometown}: Jenis daerah tempat tinggal (misalnya urban/rural)
    \item \textbf{Computer}: Kemampuan komputer
    \item \textbf{Preparation}: Waktu persiapan belajar harian
    \item \textbf{Gaming}: Waktu bermain game harian
    \item \textbf{Attendance}: Tingkat kehadiran kuliah
    \item \textbf{Job}: Status pekerjaan paruh waktu
    \item \textbf{English}: Kemampuan bahasa Inggris
    \item \textbf{Extra}: Partisipasi dalam kegiatan ekstrakurikuler
    \item \textbf{Semester}: Semester berjalan
    \item \textbf{Last}: Prestasi semester sebelumnya
    \item \textbf{Overall}: Indeks Prestasi Komulatif (IPK) keseluruhan
\end{itemize}

Dataset ini memungkinkan analisis hubungan antara variabel-variabel tersebut dan kinerja akademik mahasiswa.
