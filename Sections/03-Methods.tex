\section{Metode Penelitian}
Analisis data mahasiswa dilakukan melalui beberapa tahan terstruktur. Pertama, data mentah harus melalui tahap \textit{pra-pemrosesan} karena data mentah rentan terhadap gangguan, kerusakan, dan tidak konsisten. Data yang buruk dapat memengaruhi keakuratan dan menyebabkan prediksi yang salah, Sehingga perlu untuk meningkatkan kualitas data dengan \textit{pra-pemrosesan} \textcite{Maharana2022}. Proses \textit{pra-pemrosesan} mencakup transformasi data berikut:

\begin{itemize}
    \item \textbf{Pengodean variabel kategorikal ordinal}: Pengodean variabel kategorikal ordinal atau \textit{Integer Encoding} adalah strategi paling sederhana untuk mengkonversi data kategorikal ordinal menjadi data numerik \textcite{Pargent2022}. Sebuah bilangan bulat (\textit{integer}) diberikan kepada setiap kategori, asalkan jumlah kategori yang ada diketahui. Pengodean ini tidak menambahkan kolom baru ke data, tetapi menyiratkan urutan variabel yang mungkin tidak benar-benar ada \textcite{Potdar2017}. Sebagai contoh, pada penelitian \textcite{Prasetyawan2025}, atribut “Tingkat Pekerjaan” dengan kategori “Lokal”, “Nasional”, dan “Internasional” diubah menjadi 0, 1, dan 2 mengikuti urutannya.
    \item \textbf{\textit{One-hot encoding} variabel kategori nominal}: \textit{One-hot encoding} adalah teknik yang umum digunakan di bidang statistik dan machine learning, terutama saat menghadapi variabel kategorikal. Proses ini melibatkan representasi setiap kategori sebagai vektor biner. Dalam proses ini, vektor biner dibuat untuk setiap kategori unik, dengan semua elemen disetel ke nol kecuali yang sesuai dengan kategori pengamatan yang diberikan, yang disetel ke satu. Hal ini menghasilkan matriks vektor biner yang mewakili variabel kategorikal dalam kumpulan data (\cite{JIS2024}.
    \item 
\end{itemize}
\subsection{Method 01}
\blindtext

\subsubsection{Method 01.A}
\blindtext