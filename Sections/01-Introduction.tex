% Introduction
\section{Pendahuluan}
Dalam dunia pendidikan tinggi, pemahaman terhadap karakteristik mahasiswa menjadi aspek penting untuk meningkatkan kualitas pembelajaran. Mahasiswa memiliki latar belakang, kebiasaan belajar, serta pola perilaku yang beragam, yang secara langsung ataupun tidak langsung dapat memengaruhi capaian akademik mereka \textcite{Aljaffer2024}. Dalam konteks ini, pengelompokan atau segmentasi mahasiswa berdasarkan atribut-atribut tersebut menjadi penting agar institusi pendidikan dapat merancang pendekatan yang lebih tepat sasaran.

Salah satu metode yang dapat digunakan untuk memahami segmentasi mahasiswa adalah \textit{unsupervised learning}, khususnya \textit{clustering}. Metode ini memungkinkan peneliti untuk mengidentifikasi kelompok-kelompok mahasiswa yang memiliki kemiripan dalam berbagai aspek tanpa harus mengetahui label atau kategori sebelumnya. Dalam penelitian ini, metode \textit{K-Means Clustering} dipilih untuk mengelompokkan mahasiswa berdasarkan kebiasaan belajar, aktivitas digital (seperti waktu bermain \textit{game}), serta performa akademik mereka.

Agar analisis menjadi lebih efisien dan mudah divisualisasikan, metode \textit{Principal Component Analysis (PCA)} digunakan untuk mereduksi dimensi data tanpa kehilangan informasi penting. Dengan kombinasi metode ini, penelitian ini bertujuan untuk menggali pola-pola tersembunyi dalam data mahasiswa dan mengidentifikasi kelompok-kelompok mahasiswa yang memiliki ciri khas tertentu.

\subsection{Rumusan Masalah}

\begin{enumerate}
    \item Bagaimana segmentasi mahasiswa dapat dibentuk berdasarkan perilaku belajar dan faktor-faktor non-akademik seperti durasi bermain game?
    \item Apakah terdapat karakteristik khusus pada masing-masing segmen mahasiswa yang dapat diidentifikasi menggunakan metode clustering?
\end{enumerate}

\subsection{Tujuan Penelitian}

\begin{enumerate}
    \item Mengelompokkan mahasiswa berdasarkan atribut seperti nilai akademik, kebiasaan belajar, dan aktivitas digital menggunakan metode K-Means Clustering.
    \item Menyederhanakan dimensi data menggunakan PCA untuk visualisasi dan pemilihan fitur yang relevan.
    \item Menginterpretasikan setiap segmen mahasiswa untuk mengetahui pola perilaku yang dominan.
\end{enumerate}
