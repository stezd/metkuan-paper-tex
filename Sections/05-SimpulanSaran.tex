% Discussion
\section{Simpulan dan Saran}
\subsection{Simpulan}
Penelitian ini berhasil mengelompokkan mahasiswa ke dalam tiga klaster utama berdasarkan karakteristik akademik, perilaku, dan demografi menggunakan metode K-Means dan analisis PCA. Hasil analisis menunjukkan bahwa variabel seperti frekuensi bermain game, persiapan belajar, dan kehadiran memiliki pengaruh signifikan terhadap performa akademik mahasiswa. Klaster yang terbentuk memperlihatkan adanya perbedaan karakteristik yang cukup jelas, terutama antara mahasiswa dari departemen STEM dan non-STEM. Namun, nilai silhouette yang rendah mengindikasikan bahwa pemisahan antar klaster masih kurang optimal dan terdapat potensi tumpang tindih antar anggota klaster.

Selain itu, ditemukan bahwa mahasiswa dengan waktu bermain game yang tinggi cenderung memiliki nilai akademik yang lebih rendah, sedangkan mereka yang lebih banyak mempersiapkan diri untuk belajar cenderung memiliki performa akademik yang lebih baik. Temuan ini sejalan dengan penelitian sebelumnya yang menyoroti pentingnya perilaku belajar dan pengelolaan waktu dalam pencapaian akademik.

\subsection{Saran}
Berdasarkan hasil penelitian, beberapa saran yang dapat diberikan adalah sebagai berikut:
\begin{itemize}
    \item Institusi pendidikan disarankan untuk memberikan perhatian khusus kepada mahasiswa yang berada pada klaster dengan performa akademik rendah dan intensitas bermain game yang tinggi, misalnya melalui program pendampingan belajar atau konseling manajemen waktu.
    \item Penelitian selanjutnya dapat mempertimbangkan penggunaan metode klasterisasi lain seperti Hierarchical Clustering atau DBSCAN, serta teknik reduksi dimensi non-linier seperti t-SNE atau UMAP untuk memperoleh klaster yang lebih representatif dan mudah diinterpretasikan.
    \item Perlu dilakukan analisis lebih lanjut terhadap faktor-faktor lain yang mungkin memengaruhi performa akademik, seperti motivasi belajar, lingkungan keluarga, atau aktivitas ekstrakurikuler.
    \item Pengumpulan data dengan jumlah sampel yang lebih besar dan distribusi yang lebih merata antar departemen dapat meningkatkan validitas hasil klasterisasi.
\end{itemize}

\pagebreak
